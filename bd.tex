\documentclass[a5paper,6pt]{article}
\usepackage[utf8]{inputenc}
\usepackage{lmodern}
\usepackage[MeX]{polski}
\usepackage{geometry}
% \usepackage{amsmath}
% \usepackage{amsthm}
% \usepackage{amsfonts}
% \usepackage{cases}
% \usepackage{enumitem}
% \usepackage{mathtools}
% \usepackage{amssymb}

\newgeometry{tmargin=2cm, bmargin=2cm, lmargin=1.5cm, rmargin=1.5cm}
\setlength{\parindent}{0cm}
% \linespread{0.9}

% \newtheoremstyle{mythmstyle}%
%     {}%
%     {}%
%     {\it}%
%     {}%
%     {\bf}%
%     {}%
%     { }%
%     {\thmname{#1}. \normalfont #3 \hfill \\ }
% \theoremstyle{mythmstyle}

% % \newtheorem{example}{Przykład}[section]
% \newtheorem*{question}{Pytanie}

% \newcommand\ass{\mathrel{\overset{\makebox[0pt]{\mbox{\normalfont\tiny\sffamily zał}}}{=}}}

\newcommand{\horrule}[1]{\rule{\linewidth}{#1}}
\newcommand\tab[1][1cm]{\hspace*{#1}}

\title{Bazy Danych}
\author{Pytania egzaminacyjne}
\date{2018/2019}
\frenchspacing

\begin{document}
    \maketitle
    \tableofcontents
    \pagebreak

    \section{Architektury systemów baz danych} % (fold)
    \label{sec:architektury_systemow}

    % section architektury_systemów_baz_danych (end)

    \section{Relacyjny model danych, normalizacja relacji} % (fold)
    \label{sec:relacyjny_model_danych_normalizacja_relacji}

    \horrule{0.5pt}
    Proszę podać przykład tabeli (relacji), która jest w trzeciej postaci
    normalnej, ale nie jest w postaci normalnej Boyce’a Codd’a.\\
    \horrule{0.5pt}

    \horrule{0.5pt}
    Dla tabeli (relacji) z podanymi zależnościami funkcyjnymi proszę powiedzieć w której postaci normalnej jest ta tabela.\\
    \horrule{0.5pt}

    \horrule{0.5pt}
    Podaną tabelę należy doprowadzić do postaci normalnej Boyce’a Codd’a.\\
    \horrule{0.5pt}

    \horrule{0.5pt}
    Proszę podać przykład uzasadniający denormalizację.\\
    \horrule{0.5pt}

    \horrule{0.5pt}
    Proszę podać przykład tabeli, która jest w postaci normalnej Boyce’a Codd’a,
    a w której jest redundancja.\\
    \horrule{0.5pt}

    \horrule{0.5pt}
    Proszę podać przykład tabeli, która jest w 5NF, ale jest w niej
    redundancja.\\
    \horrule{0.5pt}

    \horrule{0.5pt}
    Proszę \underline{przedstawić} algorytm doprowadzenia relacji do
    \textbf{postaci normalnej Boyce’a Codd’a}.\\
    \horrule{0.5pt}

    \begin{itemize}
        \item Szukamy wszystkich nietrywialnych, pełnych zależności funkcyjnych,
              które naruszają warunek BCNF, tzn. lewa strona zalezności
              funkcyjnej nie jest nadkluczem.
        \item Bierzemy jedną z takich zależności funkcyjnych $A \rightarrow B$
              (obojetnie którą, algorym jest niederministyczny).
        \item Dzielimy schemat relacji na dwa nierozłączne podzbiory:
              jeden zawierający wszystkie atrybuty wystepujące w zależności (*)
              naruszającej BCNF,
              drugi zawierający atrybuty z lewej storny rozważanej zależności
              (*) oraz atrubuty nie wystąpujące ani z lewej ani z prawej strony
              tej zależności.
        \item Strategie stosujemy do relacji powstałych w wyniku dekompozycji
              do chwili, gdy wszystkie relacje są w BCNF.
    \end{itemize}

    % section relacyjny_model_danych_normalizacja_relacji (end)

    \section{Model ER} % (fold)
    \label{sec:model_er}

    \horrule{0.5pt}
    Proszę przedstawić przykład diagramu ER w notacji Barkera, zawierającego
    dwie encje i związek między nimi (związek bez własnych atrybutów). Diagram
    powinien być taki, że po jego transformacji do modelu relacyjnego otrzymamy
    trzy relacje (trzy tabele).\\
    \horrule{0.5pt}

    \horrule{0.5pt}
    Proszę \underline{omówić} trzy schematy transformacji hierarchii encji do modelu relacyjnego.\\
    \horrule{0.5pt}

    \begin{enumerate}
        \item Utworzenie jednej tabeli ze wszystkimi atrybutami i kluczami
              obcymi, tj. wspólnymi i specyficznymi dla podencji.
        \item Utworzenie osobej tabeli dla każdej podencji.
        \item Utworzenie osobnej tabeli na atrybuty wspólne i osobnej tabeli
              dla każdej podencji.\\
              Tabele powstałe z podencji zawierają klucz podstawiowy i atrybuty
              specyficzne, tabela wspólna i tabele powstałe z podencji są
              połączone ograniczeniami referencyjnymi.
    \end{enumerate}

    % section model_er (end)

    \section{Transakcje} % (fold)
    \label{sec:transakcje}

    \subsection{Właściwiości (ACID)} % (fold)
    \label{sub:wlasciwiosci}

    \horrule{0.5pt}
     Proszę omówić własności \textbf{ACID} transakcji. W jaki sposób
          implementowane są własności \textbf{A i D}? Proszę podać jak
          wykorzystywany jest \textbf{dziennik transakcji} oraz co to jest
          strategia \textbf{No-Fix/No-Flush} i jak wpływa ona na sposoby
          odtwarzania systemu po awarii.\\
    \horrule{0.5pt}

    \textbf{WŁASNOŚCI ACID}
    \begin{itemize}
        \item \textbf{Atomicity} (atomowość, niepodzielność)
        \begin{itemize}
            \item Transakcja jest niepodzielną jednostką przetwarzania, musi
                  być wykonywana w całości lub wcale.
        \end{itemize}

        \item \textbf{Consistency} (spójność)
        \begin{itemize}
            \item Po wykonaniu transakcji baza danych musi być w stanie spójnym,
                  tj. muszą zostać zachowane wszystkie więzy narzucone na dane.
        \end{itemize}

        \item \textbf{Isolation} (separacja, izolacja)
        \begin{itemize}
            \item Transakcja powinna wyglądać tak, jakby była wykonywana w
                  izolacji od innych transakcji.
        \end{itemize}

        \item \textbf{Durability} (trwałość)
        \begin{itemize}
            \item Po zatwierdzeniu transakcji jej efekty muszą być trwałe w
                  systemie, nawet jeśli nastąpi uszkodzenie systemu natychmiast
                  po zatwierdzeniu.
        \end{itemize}

    \end{itemize}

    \textbf{IMPLEMENTACJA A I D}
    \begin{itemize}
        \item Za odtwarzanie i w pewnym sensie za \textbf{atomowość, trwałość}
              oraz spójność, jest odpowiedzialny \textbf{moduł zarządzania
              odtwarzaniem} \textit{(recovery-management component).}

        \item Modyfikacja danych następuje w buforze w pamięci RAM. Buforem
              zarządza specjalny menadżer (zarządca). W pewnym momencie zarządca
              bufora musi skopiować nową zawartość bloku z powrotem na dysk.
    \end{itemize}

    \textbf{STRATEGIA NO-FIX/NO-FLUSH}
    \begin{itemize}
        \item Stosowana w większości relacyjnych systemów baz zdanych
              wykorzystujących dzienniki transakcji.

        \item \textbf{NO-FIX}
        \begin{itemize}
            \item Blok skopiowany do RAM \textbf{może} być skopiowany lub
                  przeniesiony z powrotem na dysk zanim transakcja, która ten
                  blok zmodyfikowała się skończy.
        \end{itemize}

        \item \textbf{NO-FLUSH}
        \begin{itemize}
            \item Na końcu transakcji \textbf{nie ma obowiązku}
                  zsynchronizowania zmienionych przez tę transakcję bloków
                  z dyskiem.
            \item Synchronizacja może być wykonana później.
            \item Zwiększa wydajność.
            \item Na tradycyjnych dyskach HDD operacje kopiowania bloków mogą
                  być grupowane.
        \end{itemize}
    \end{itemize}

    \textbf{WPŁYW NA SPOSOBY ODTWARZANIA SYSTEMU PO AWARII}
    \begin{itemize}
        \item \textbf{No-Flush} oznacza, że nie mamy gwarancji, że natychmiast
              po zakończeniu transakcji na dysku znajdą się zmienione dane.
        \item Trwałość transakcji w większości systemów zapewniają
              \textbf{dzienniki transakcji}.
        \item \textbf{No-Fix} oznacza, że może się zdarzyć, że blok zmieniony
              przez transakcje zostanie skopiowany na dysk, zanim transakcja
              zakończy się.
        \item Synchronizacja buforów z RAM z dyskiem może być realizowane
              cyklicznie w ramach \textbf{punktów kontrolnych}
              \textit{(control point, check point)}.
        \item \textbf{Punkty kontrolne} ułatwiają \textbf{odtwarzanie systemu
              po awarii}, kiedy dyski z danymi i z dziennikiem transakcji nie
              zostały uszkodzone.

        \item \textbf{Odtwarzanie po awarii systemu (RECOVERY)}
        \begin{itemize}
            \item redo
            \item undo
        \end{itemize}

        \item \textbf{Odtwarzanie po awarii dysków z danymi}
        \begin{itemize}
            \item RESTORE (przywracanie plików z kopii zapasowych)
            \item RECOVERY
        \end{itemize}
    \end{itemize}

    \textbf{STOSOWANIE DZIENNIKA TRANSAKCJI}
    \begin{itemize}
        \item \textbf{Dziennik transakcji} jest plikiem na dysku, zawierającym
              \textbf{informację o wszystkich wprowadzonych} przez transakcję
              \textbf{zmianach}.
        \item Transakcja \textbf{nie jest uznana} za zakończoną, jeśli fizycznie
              na dysku w pliku dziennika nie znajdą się wpisy opisujące
              wszystkie zmiany oraz informacja o zatwierdzeniu transakcji.
        \item Wpis (rekord) w dzienniku może zawierać znacznik transakcji, starą
              i nową wartość zmienianego elementu, informację o rodzaju operacji,
              może zawierać informację o tzw. operacji kompensującej. Są też
              wpisy dotyczące rozpoczęcia transakcji i jej zatwierdzenia, w
              pewnych systemach także wpisy dotyczące operacji odczytu.
        \item Dzięki dziennikowi można powtórzyć te operacje, których efekty nie
              zostały jeszcze zapisane na dysku, mimo że operacja została
              zatwierdzona (no-flush).
        \item \textbf{W przypadku odtwarzania po awarii} może być wymagane
              \textbf{powtórzenie} \textit{(redo)} niektórych operacji
              (zatwierdzone) i \textbf{wycofanie} \textit{(undo)} innych
              (niezatwierdzone).
    \end{itemize}

    \horrule{0.5pt}

    % subsection właściwiości (end)

    \subsection{Harmonogramy, szeregowalność konfliktowa i perspektywiczna} % (fold)
    \label{sub:harmonogramy_szeregowalnosc_konfliktowa_i_perspektywiczna}

    \horrule{0.5pt}
    Proszę podać definicje \textbf{harmonogramu szeregowalnego, szeregowalnego
    konfliktowo i szeregowalnego perspektywicznie}. Jakie znaczenie w praktyce
    ma pojęcie \textbf{szeregowalności konfliktowej?}\\
    \horrule{0.5pt}

    \textbf{HARMONOGRAM SZEREGOWALNY}
    \begin{itemize}
        \item \textbf{Jeśli jego wpływ na stan bazy danych jest taki sam jak
              pewnego harmonogramu szeregowego, niezależnie od stanu
              początkowego tej bazy danych}.
        \item Harmonogramy są \textbf{równoważne co do wyniku}
              \textit{(result equivalent)}, jeżeli dają ten sam stan bazy danych
              bez względu na początkowy stan bazy.
    \end{itemize}

    \textbf{HARMONOGRAM SZEREGOWALNY KONFLIKTOWO}
    \textit{(conflict serializable)}
    \begin{itemize}
        \item Harmonogramy są \textbf{równoważne konfliktowo}
              \textit{(conflict equivalent)} jeżeli kolejność wszystkich
              operacji konfliktowych jest w nich taka sama.
        \item Harmonogram \textbf{S} jest \underline{szeregowalny konfliktowo},
        jeżeli jest on \underline{równoważny} \underline{konfliktowo} z pewnym
        szeregowym harmonogramem \textbf{S'}.
        \item W takim przypadku możemy zamieniać kolejność niekonfliktowych
        operacji w \textbf{S} do momentu, aż utworzony zostanie równoważny
        harmonogram szeregowy \textbf{S'}.
    \end{itemize}

    \textbf{HARMONOGRAM SZEREGOWALNY PERSPEKTYWICZNIE}
    \textit{(view serializability)}
    \begin{itemize}
        \item Jest on \textbf{równoważny perspektywicznie} pewnemu
        harmonogramowi szeregowemu.
        \item \textbf{Równoważność perspektywiczna}:
        \begin{itemize}
            \item Harmonogram \textbf{S i S'} zwierają te same instrukcje i dla
                  każdego elementu danych \textbf{Q}:
            \begin{enumerate}
                \item $T_k$ jest transakcją, która czyta \textbf{Q} jako
                      pierwsza $=>$ $T_k$ musi być transakcją, która czyta
                      \textbf{Q} jako pierwsza.
                \item $T_i$ czyta \textbf{Q} zapisany przez $T_j => T_i$ czyta
                      \textbf{Q} zapisany przez $T_j$
                \item $T_m$ jest ostatnią transakcją, która zapisuje \textbf{Q}
                      $=> T_m$ jest ostatnią transakcją, która zapisuje
                      \textbf{Q}
            \end{enumerate}
        \end{itemize}
        \item Oznacza to samo co \textbf{szeregowalność konfliktowa, jeśli}
              założymy ograniczenie co do operacji \textbf{zapisów} we
              wszystkich transakcjach harmonogramu.
        \begin{itemize}
            \item Każda operacja \texttt{WRITE[x]} jest poprzedzona operacją
                  \texttt{READ[x]}
            \item Wartość zapisana przez \texttt{WRITE[x]} zależy tylko od
                  wartości elementu \texttt{x} odczytanej przez operacje
                  \texttt{READ[x]} (jest pewną nie stałą funkcją tylko elementu
                  \texttt{x}, nie zależy od wartości innych elementów.)
        \end{itemize}

        \item Szeregowalność perspektywiczna zapenia \textbf{spójność} bazy
              danych, ponieważ powoduje, że wyniki harmonogramu są takie same
              jak wyniki pewnego harmonogamu szeregowego.
    \end{itemize}

    \textbf{ZNACZENIE W PRAKTYCE SZEREGOWALNOŚCI KONFLIKTOWEJ}
    \begin{itemize}
        \item Zapewnia \textbf{spójność} bazy danych.
    \end{itemize}

    \horrule{0.5pt}
    Proszę podać przykłady harmonogramów szeregowalnych nie szeregowych.\\
    \horrule{0.5pt}

    \horrule{0.5pt}
    Proszę podać przykłady harmonogramów szeregowalnych konfliktowo i takich,
    które nie są szeregowalne konfliktowo.\\
    \horrule{0.5pt}

    % subsection harmonogramy_szeregowalność_konfliktowa_i_perspektywiczna (end)

    \subsection{Poziomy izolacji transakcji} % (fold)
    \label{sub:poziomy_izolacji_transakcji}

    \horrule{0.5pt}
    Proszę omówić poziom izolacji transakcji wybrany przez egzaminatora.\\
    \horrule{0.5pt}


    % subsection poziomy_izolacji_transakcji (end)

    \subsection{Sterowanie współbieżnymi transakcjami w oparciu o blokady} % (fold)
    \label{sub:sterowanie_wspolbieznymi_blokady}

    % subsection sterowanie_współbieżnymi_transakcjami_w_oparciu_o_blokady (end)

    \subsection{Sterowanie współbieżnymi transakcjami z wykorzystaniem
    wielowersyjności i blokad} % (fold)
    \label{sub:sterowanie_wspolbieznymi_wielowier}

    \horrule{0.5pt}
    Dla przedstawionego harmonogramu proszę podać jak będzie wyglądać sterowanie
    współbieżnością w wybranym przez egzaminatora poziomie izolacji
    transakcji.\\
    \horrule{0.5pt}

    \horrule{0.5pt}
    Proszę omówić wybrane przez egzaminatora problemy związane ze współbieżnym
    wykonaniem transakcji.\\
    \horrule{0.5pt}

    % subsection sterowanie_współbieżnymi_transakcjami (end)

    \subsection{Zakleszczenia} % (fold)
    \label{sub:zakleszczenia}

    \horrule{0.5pt}
    Proszę napisać przykładowy harmonogram, który doprowadzi do zakleszczenia.
    Jak mogą być wykrywane zakleszczenia?\\
    \horrule{0.5pt}

    % subsection zakleszczenia (end)

    \subsection{Kursory, sterowanie współbieżnością w kursorach} % (fold)
    \label{sub:kursory_sterowanie_wspolbieznoscia_w_kursorach}

    \horrule{0.5pt}
    Proszę omówić, jak wygląda sterowanie współbieżnością w kursorach w systemie
    Microsoft SQL Server.\\
    \horrule{0.5pt}

    % subsection kursory_sterowanie_współbieżnością_w_kursorach (end)

    % section transakcje (end)

    \section{Język SQL} % (fold)
    \label{sec:jezyk_sql}

    \horrule{0.5pt}
    Proszę napisać zdanie w języku SQL, które zrealizuje cel podany przez egzaminatora.\\
    \horrule{0.5pt}


    % section język_sql (end)

    \section{Procedury, funkcje i wyzwalacze} % (fold)
    \label{sec:procedury_funkcje_i_wyzwalacze}

    \horrule{0.5pt}
    Proszę napisać procedurę w języku Transact SQL, funkcjonalność procedury
    poda egzaminator.\\
    \horrule{0.5pt}

    \horrule{0.5pt}
    Proszę napisać funkcję skalarną w języku Transact SQL, opisaną przez egzaminatora.\\
    \horrule{0.5pt}

    \horrule{0.5pt}
    Proszę napisać w języku Transact SQL opisaną przez egzaminatora funkcję zwracającą zestaw rekordów.\\
    \horrule{0.5pt}

    \horrule{0.5pt}
    Proszę napisać w języku Transact SQL wyzwalacz, który zrealizuje zadaną
    przez egzaminatora funkcjonalność.\\
    \horrule{0.5pt}

    % section procedury_funkcje_i_wyzwalacze (end)

    \section{Indeksy, typy indeksów, statystyki, wykorzystanie przez
             optymalizatory kwerend} % (fold)
    \label{sec:indeksy}

    \horrule{0.5pt}
    Proszę omówić budowę indeksu typu drzewo B+. Proszę podać wersje tego
    indeksu (w systemie Microsoft SQL Server: clustered i non-clustered, w
    Oracle IOT).\\
    \horrule{0.5pt}

    % section indeksy (end)

    \section{Budowa fizyczna baz danych - macierze RAID} % (fold)
    \label{sec:budowa_fizyczna_baz_danych_macierze_raid}

    \horrule{0.5pt}
    Proszę omówić macierze RAID (wybrany przez egzaminatora poziom).\\
    \horrule{0.5pt}

    % section budowa_fizyczna_baz_danych_macierze_raid (end)

    \section{Charakterystyka baz danych NoSQL} % (fold)
    \label{sec:charakterystyka_baz_danych_nosql}

    \horrule{0.5pt}
    Proszę powiedzieć co oznacza termin skalowanie poziome w systemach baz
    danych i jak realizowane jest skalowanie poziome w bazach danych NoSQL.\\
    \horrule{0.5pt}

    \horrule{0.5pt}
    Proszę podać główne typy baz danych NoSQL. Proszę podać przykłady dokumentów
    w formacie JSON. Jak dokumenty są przechowywane w systemach NoSQL
    zawierających wiele węzłów (komputerów) połączonych siecią komputerową?
    Czy łączenie danych z różnych dokumentów odbywa się na ogół w bazie danych
    (jak operacja JOIN w bazach relacyjnych), czy w aplikacji? Czy w bazach
    dokumentowych należy unikać redundancji tak, jak w systemach relacyjnych?\\
    \horrule{0.5pt}

    % section charakterystyka_baz_danych_nosql (end)

\end{document}
