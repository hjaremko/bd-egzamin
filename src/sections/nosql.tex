\section{Charakterystyka baz danych NoSQL}
\label{sec:charakterystyka_baz_danych_nosql}

\horrule{0.5pt}
Proszę \underline{powiedzieć co oznacza} termin skalowanie \textbf{poziome}
w systemach baz danych i \textbf{jak realizowane jest} skalowanie poziome
w bazach danych NoSQL.\\
\horrule{0.5pt}

\vskip 0.5cm

\textbf{SKALOWANIE POZIOME}
\begin{itemize}
    \item Jest to dodawanie dodatkowych "węzłów" do systemu, na przykład
          dodawanie nowego komputera do systemu.
\end{itemize}

\textbf{SKALOWANIE PIONOWE}
\begin{itemize}
    \item Jest to dodawanie dodatkowych zasobów istniejącym "węzłom" w
    systemie, na przykład dodawanie RAM komputerowi w systemie.
\end{itemize}

\textbf{REALIZACJA SKALOWANIA POZIOMEGO}
\begin{itemize}
    \item Rezygnacja z części funkcjonalności relacyjnych baz danych,
          brak relacji, transakcji, nie są zachowane własności ACID.
\end{itemize}

\horrule{0.5pt}
Proszę \underline{podać główne typy} baz danych \textbf{NoSQL}.\\
Proszę \underline{podać przykłady dokumentów} w formacie \textbf{JSON}.\\
\underline{Jak} dokumenty są przechowywane w systemach \textbf{NoSQL}
zawierających wiele węzłów (komputerów) połączonych siecią komputerową?\\
Czy \underline{łączenie danych} z różnych dokumentów odbywa się na ogół w
bazie danych
(jak operacja \texttt{JOIN} w bazach relacyjnych), czy w aplikacji?\\
Czy w bazach \textbf{dokumentowych} należy unikać redundancji tak, jak w
systemach relacyjnych?\\
\horrule{0.5pt}

\vskip 0.5cm

\textbf{GŁÓWNE TYPY}
\begin{itemize}
    \item Klucz-Wartość
    \item Hierarchiczna struktura klucz-wartość
    \item Dokumentowe
    \item Grafowe
\end{itemize}

\pagebreak

\textbf{PRZYKŁAD JSON}
\begin{minted}[mathescape,
           linenos,
           numbersep=5pt,
           gobble=2,
           frame=lines,
           framesep=2mm]{json}
  {
    "menu":
    {
     "id": "file",
     "value": "File",
     "popup":
     {
       "menuitem":
       [
         {"value": "New", "onclick": "CreateNewDoc()"},
         {"value": "Open", "onclick": "OpenDoc()"},
         {"value": "Close", "onclick": "CloseDoc()"}
       ]
     }
    }
  }
\end{minted}

\textbf{PRZECHOWYWANIE DOKUMENTÓW}
\begin{itemize}
    \item Dokumenty rozrzucone po węzłach, każdy z węzłów czyta i zapisuje
          równolegle.
\end{itemize}

\textbf{ŁĄCZENIE DOKUMENTÓW}
\begin{itemize}
    \item Odbywa sie w aplikacji.
\end{itemize}

\textbf{UNIKANIE REDUNDANCJI}
\begin{itemize}
    \item Nie należy unikać redundancji.
    \item Liczy się szybkość dostępu do danych, często przez wielu
          użytkowników na raz.
\end{itemize}
