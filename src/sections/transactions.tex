\section{Transakcje}
\label{sec:transakcje}

\subsection{Właściwiości (ACID)}
\label{subsec:wlasciwiosci}

\horrule{0.5pt}
Proszę omówić własności \textbf{ACID} transakcji. W jaki sposób
implementowane są własności \textbf{A i D}? Proszę podać jak
wykorzystywany jest \textbf{dziennik transakcji} oraz co to jest
strategia \textbf{No-Fix/No-Flush} i jak wpływa ona na sposoby
odtwarzania systemu po awarii.\\
\horrule{0.5pt}

\textbf{WŁASNOŚCI ACID}
\begin{itemize}
    \item \textbf{Atomicity} \textit{(atomowość, niepodzielność)}
    \begin{itemize}
        \item Transakcja jest niepodzielną jednostką przetwarzania, musi
        być wykonywana w całości lub wcale.
    \end{itemize}

    \item \textbf{Consistency} \textit{(spójność)}
    \begin{itemize}
        \item Po wykonaniu transakcji baza danych musi być w stanie spójnym,
        tj. muszą zostać zachowane wszystkie więzy narzucone na dane.
    \end{itemize}

    \item \textbf{Isolation} \textit{(separacja, izolacja)}
    \begin{itemize}
        \item Transakcja powinna wyglądać tak, jakby była wykonywana w
        izolacji od innych transakcji.
    \end{itemize}

    \item \textbf{Durability} \textit{(trwałość)}
    \begin{itemize}
        \item Po zatwierdzeniu transakcji jej efekty muszą być trwałe w
        systemie, nawet jeśli nastąpi uszkodzenie systemu natychmiast
        po zatwierdzeniu.
    \end{itemize}

\end{itemize}

\textbf{IMPLEMENTACJA A I D}
\begin{itemize}
    \item Za odtwarzanie i w pewnym sensie za \textbf{atomowość, trwałość}
    oraz spójność, jest odpowiedzialny \textbf{moduł zarządzania
    odtwarzaniem} \textit{(recovery-management component).}

    \item Modyfikacja danych następuje w buforze w pamięci RAM. Buforem
    zarządza specjalny menadżer \textit{(zarządca)}.
    W pewnym momencie zarządca bufora musi skopiować nową zawartość
    bloku z powrotem na dysk.
\end{itemize}

\pagebreak

\textbf{STRATEGIA NO-FIX/NO-FLUSH}
\begin{itemize}
    \item Stosowana w większości relacyjnych systemów baz zdanych
    wykorzystujących dzienniki transakcji.

    \item \textbf{NO-FIX}
    \begin{itemize}
        \item Blok skopiowany do RAM \textbf{może} być skopiowany lub
        przeniesiony z powrotem na dysk zanim transakcja, która ten
        blok zmodyfikowała się skończy.
    \end{itemize}

    \item \textbf{NO-FLUSH}
    \begin{itemize}
        \item Na końcu transakcji \textbf{nie ma obowiązku}
        zsynchronizowania zmienionych przez tę transakcję bloków
        z dyskiem.
        \item Synchronizacja może być wykonana później.
        \item \underline{Zwiększa wydajność}.
        \item Na tradycyjnych dyskach HDD operacje kopiowania bloków mogą
        być grupowane.
    \end{itemize}
\end{itemize}

\textbf{WPŁYW NA SPOSOBY ODTWARZANIA SYSTEMU PO AWARII}
\begin{itemize}
    \item \textbf{No-Flush} oznacza, że nie mamy gwarancji, że natychmiast
    po zakończeniu transakcji na dysku znajdą się zmienione dane.
    \item Trwałość transakcji w większości systemów zapewniają
    \textbf{dzienniki transakcji}.
    \item \textbf{No-Fix} oznacza, że może się zdarzyć, że blok zmieniony
    przez transakcje zostanie skopiowany na dysk, zanim transakcja
    zakończy się.
    \item Synchronizacja buforów z RAM z dyskiem może być realizowane
    cyklicznie w ramach \textbf{punktów kontrolnych}
    \textit{(control point, check point)}.
    \item \textbf{Punkty kontrolne} ułatwiają \textbf{odtwarzanie systemu
    po awarii}, kiedy dyski z danymi i z dziennikiem transakcji nie
    zostały uszkodzone.

    \pagebreak

    \item \textbf{Odtwarzanie po awarii systemu (RECOVERY)}
    \begin{itemize}
        \item redo
        \item undo
    \end{itemize}

    \item \textbf{Odtwarzanie po awarii dysków z danymi}
    \begin{itemize}
        \item RESTORE \textit{(przywracanie plików z kopii zapasowych)}
        \item RECOVERY
    \end{itemize}
\end{itemize}

\textbf{STOSOWANIE DZIENNIKA TRANSAKCJI}
\begin{itemize}
    \item \textbf{Dziennik transakcji} jest plikiem na dysku, zawierającym
    \textbf{informację o\\ wszystkich wprowadzonych} przez transakcję
    \textbf{zmianach}.
    \item Transakcja \textbf{nie jest uznana} za zakończoną, jeśli fizycznie
    na dysku w pliku dziennika nie znajdą się \textbf{wpisy opisujące
    \underline{wszystkie} zmiany} oraz \textbf{informacja o
    zatwierdzeniu} transakcji.
    \item Wpis (rekord) w dzienniku może zawierać znacznik transakcji, starą
    i nową wartość zmienianego elementu, informację o rodzaju
    operacji, może zawierać informację o tzw. operacji kompensującej.
    Są też wpisy dotyczące rozpoczęcia transakcji i jej
    zatwierdzenia, w pewnych systemach także wpisy dotyczące operacji
    odczytu.
    \item Dzięki dziennikowi można powtórzyć te operacje, których efekty nie
    zostały jeszcze zapisane na dysku, mimo że operacja została
    zatwierdzona (no-flush).
    \item \textbf{W przypadku odtwarzania po awarii} może być wymagane
    \textbf{powtórzenie} \textit{(redo)} niektórych operacji
    (zatwierdzone, ze względu na \textit{\textbf{No-Flush}} mogły\\
    jeszcze nie zostać zapisane) i \textbf{wycofanie} \textit{(undo)}
    innych (niezatwierdzone, ze względu na \textit{\textbf{No-Fix}}
    mogły już się zapisać).
\end{itemize}

% subsection właściwiości (end)

\pagebreak

\subsection{Harmonogramy, szeregowalność konfliktowa i perspektywiczna}
\label{sub:harmonogramy_szeregowalnosc_konfliktowa_i_perspektywiczna}

\horrule{0.5pt}
Proszę podać definicje \textbf{harmonogramu szeregowalnego, szeregowalnego
konfliktowo i szeregowalnego perspektywicznie}. Jakie znaczenie w praktyce
ma pojęcie \textbf{szeregowalności konfliktowej?}\\
\horrule{0.5pt}

\vskip 0.5cm

\textbf{HARMONOGRAM SZEREGOWALNY}
\begin{itemize}
    \item \textbf{Jeśli jego wpływ na stan bazy danych jest taki sam jak
    pewnego harmonogramu szeregowego, niezależnie od stanu
    początkowego tej bazy danych}.
    \item Harmonogramy są \textbf{równoważne co do wyniku}
    \textit{(result equivalent)}, jeżeli dają ten sam stan bazy danych
    bez względu na początkowy stan bazy.
\end{itemize}

\textbf{HARMONOGRAM SZEREGOWALNY KONFLIKTOWO}
\textit{(conflict serializable)}
\begin{itemize}
    \item Harmonogramy są \textbf{równoważne konfliktowo}
    \textit{(conflict equivalent)} jeżeli kolejność wszystkich
    operacji konfliktowych jest w nich taka sama.
    \item Dwie operacje są w stanie \textbf{konfliktu}, jeśli:
    \begin{itemize}
        \item Należą do różnych transakcji.
        \item Uzyskują dostep do tego samego elementu.
        \item Przynajmnej jedna z nich jest operacją zapisu.
    \end{itemize}

    \item Harmonogram \textbf{S} jest \underline{szeregowalny konfliktowo},
    jeżeli jest on \underline{równoważny} \underline{konfliktowo} z pewnym
    szeregowym harmonogramem \textbf{S'}.
    \item W takim przypadku możemy zamieniać kolejność niekonfliktowych
    operacji w \textbf{S} do momentu, aż utworzony zostanie równoważny
    harmonogram szeregowy \textbf{S'}.
\end{itemize}

\pagebreak

\textbf{HARMONOGRAM SZEREGOWALNY PERSPEKTYWICZNIE}\\
\textit{(view serializability)}
\begin{itemize}
    \item Jest on \textbf{równoważny perspektywicznie} pewnemu
    harmonogramowi szeregowemu.
    \item \textbf{Równoważność perspektywiczna}:
    \begin{itemize}
        \item Harmonogram \textbf{S i S'} zwierają te same instrukcje i dla
        każdego elementu danych \textbf{Q}:
    \end{itemize}

    \begin{adjustbox}{width=\columnwidth,center}
        \begin{tabular}{|l|l|}
            \hline
            $S$ & $S'$ \\
            \hline
            $T_k$ jest pierwszą transakcją, która czyta \textbf{Q} &
            $T_k$ musi być pierwszą transakcją, która czyta \textbf{Q} \\

            $T_i$ czyta \textbf{Q} zapisany przez $T_j$ &
            $T_i$ czyta \textbf{Q} zapisany przez $T_j$\\

            $T_m$ jest ostatnią transakcją, która zapisuje \textbf{Q} &
            $T_m$ jest ostatnią transakcją, która zapisuje \textbf{Q}\\
            \hline
        \end{tabular}
    \end{adjustbox}

    \item Oznacza to samo co \textbf{szeregowalność konfliktowa, jeśli}
    założymy ograniczenie co do operacji \textbf{zapisów} we
    wszystkich transakcjach harmonogramu.
    \begin{itemize}
        \item Każda operacja \texttt{WRITE[x]} jest poprzedzona operacją
        \texttt{READ[x]}
        \item Wartość zapisana przez \texttt{WRITE[x]} zależy tylko od
        wartości elementu \texttt{x} odczytanej przez operacje
        \texttt{READ[x]} \textit{(jest pewną nie stałą funkcją tylko elementu
        \texttt{x}, nie zależy od wartości innych elementów)}.
    \end{itemize}

    \item Szeregowalność perspektywiczna zapenia \textbf{spójność} bazy
    danych, ponieważ powoduje, że wyniki harmonogramu są takie same
    jak wyniki pewnego harmonogamu szeregowego.
\end{itemize}

\textbf{ZNACZENIE W PRAKTYCE SZEREGOWALNOŚCI KONFLIKTOWEJ}
\begin{itemize}
    \item Zapewnia \textbf{spójność} bazy danych.
\end{itemize}

\pagebreak

\horrule{0.5pt}
Proszę \underline{podać przykłady} harmonogramów \textbf{szeregowalnych}
\underline{nie szeregowych}.\\
\horrule{0.5pt}

\begin{center}
    \begin{tabular}{|p{4cm}|p{4cm}|}
        \hline
        \textbf{$T_1$} & \textbf{$T_2$} \\
        \hline
        \texttt{read(A)} & \texttt{}\\
        \texttt{A := A - 50} & \texttt{}\\
        \texttt{write(A)} & \texttt{}\\
        \texttt{} & \texttt{read(A)}\\
        \texttt{} & \texttt{temp := A * 0.1}\\
        \texttt{} & \texttt{A := A - temp}\\
        \texttt{} & \texttt{write(A)}\\
        \texttt{read(B)} & \texttt{}\\
        \texttt{B := B + 50} & \texttt{}\\
        \texttt{write(B)} & \texttt{}\\
        \texttt{} & \texttt{read(B)}\\
        \texttt{} & \texttt{B := B + temp}\\
        \texttt{} & \texttt{write(B)}\\
        \hline
    \end{tabular}
\end{center}

\pagebreak

\horrule{0.5pt}
Proszę \underline{podać przykłady} harmonogramów
\textbf{szeregowalnych konfliktowo} i takich, które \textbf{nie są}
szeregowalne konfliktowo.\\
\horrule{0.5pt}

\textbf{HARMONOGRAMY SZERGOWALNE KONFLIKTOWO}

\begin{center}
    \begin{tabular}{|p{4cm}|p{4cm}|}
        \hline
        \textbf{$T_1$} & \textbf{$T_2$} \\
        \hline
        \texttt{read(A)} & \texttt{}\\
        \texttt{write(A)} & \texttt{}\\
        \texttt{} & \texttt{read(A)}\\
        \texttt{} & \texttt{write(A)}\\
        \texttt{read(B)} & \texttt{}\\
        \texttt{write(B)} & \texttt{}\\
        \texttt{} & \texttt{read(B)}\\
        \texttt{} & \texttt{write(B)}\\
        \hline
    \end{tabular}
\end{center}

\textbf{HARMONOGRAMY \underline{NIE}SZERGOWALNE KONFLIKTOWO}

\begin{center}
    \begin{tabular}{|p{4cm}|p{4cm}|}
        \hline
        \textbf{$T_1$} & \textbf{$T_2$} \\
        \hline
        \texttt{read(A)} & \texttt{}\\
        \texttt{A := A - 50} & \texttt{}\\
        \texttt{write(A)} & \texttt{}\\
        \texttt{} & \texttt{read(B)}\\
        \texttt{} & \texttt{B := 5 - 10}\\
        \texttt{} & \texttt{write(B)}\\
        \texttt{read(B)} & \texttt{}\\
        \texttt{B := 5 + 50} & \texttt{}\\
        \texttt{write(B)} & \texttt{}\\
        \texttt{} & \texttt{read(A)}\\
        \texttt{} & \texttt{A := A + 10}\\
        \texttt{} & \texttt{write(A)}\\
        \hline
    \end{tabular}
\end{center}

\pagebreak

\subsection{Poziomy izolacji transakcji}
\label{subsec:poziomy_izolacji_transakcji}

\horrule{0.5pt}
Proszę \underline{omówić} \textbf{poziom izolacji} transakcji wybrany przez
egzaminatora.\\
\horrule{0.5pt}

\vskip 1cm

% \begin{table}
\begin{adjustbox}{width=\columnwidth,center}
    % \begin{center}
    \begin{tabular}{|p{3cm}|p{1cm}|p{1cm}|p{2cm}|p{1cm}|p{1.3cm}|p{1.3cm}|}
        \hline
        \vskip 0.1cm
        \textbf{Poziom izolacji} &
        \textbf{P0 Dirty Write} &
        \textbf{P1 Dirty Read} &
        \textbf{P2 \newline Non-repeatable read} &
        \textbf{P3 Phantoms} &
        \textbf{Blokady X} &
        \textbf{Blokady S}\\
        \hline
        Locking READ \newline UNCOMMITTED &
        \cellcolor{red!25} NIE &
        \cellcolor{green!25} TAK &
        \cellcolor{green!25} TAK &
        \cellcolor{green!25} TAK &
        \cellcolor{yellow!25} TAK, długie &
        Nie ma\\
        \hline
        Locking READ \newline COMMITTED &
        \cellcolor{red!25} NIE &
        \cellcolor{red!25} NIE &
        \cellcolor{green!25} TAK &
        \cellcolor{green!25} TAK &
        \cellcolor{yellow!25} TAK, długie &
        \cellcolor{cyan!15} TAK, krótkie\\
        \hline
        Locking REPEATABLE READ &
        \cellcolor{red!25} NIE &
        \cellcolor{red!25} NIE &
        \cellcolor{red!25} NIE &
        \cellcolor{green!25} TAK &
        \cellcolor{yellow!25} TAK, długie &
        \cellcolor{yellow!25} TAK, długie\\
        \hline
        Locking \newline SERIALIZABLE &
        \cellcolor{red!25} NIE &
        \cellcolor{red!25} NIE &
        \cellcolor{red!25} NIE &
        \cellcolor{red!25} NIE &
        \cellcolor{yellow!25} TAK, długie &
        \cellcolor{blue!25} TAK, długie, predykatowe\\
        \hline

    \end{tabular}
    % \end{center}
\end{adjustbox}
% \end{table}
\vskip 1cm

% \textbf{BLOKADY} \\
\textbf{CURSOR STABILITY}
\begin{itemize}
    \item READ COMMITTED « Cursor Stability « REPEATABLE READ
    \item Pewne rozszerzenie Locking READ COMMITTED.
    \item Dodaje sie operacje \texttt{READ\_CURSOR}
    \textit{(pobierz wiersz)} dla instrukcji \texttt{FETCH}.
    \item Blokada (\texttt{S} lub nowy typ do odczytu \textit{scroll lock})
    bedzie utrzymywana do\\ chwili przejścia do innego wiersza lub
    zamknięcia kursora.
    \item Aktualizacja wiersza przez kursor - operacja
    \texttt{WRITE\_CURSOR} powoduje\\ założenie na ten wiersz
    \textbf{blokady \texttt{X}} utrzymywanej do końca transakcji.
    \item \textbf{Eliminuje problem P4C}.
\end{itemize}

\textbf{SNAPSHOT ISOLATION} \textit{(First-commiter-wins)}
\begin{itemize}
    \item Transakcja czyta dane (zatwierdzone) z chwili swojego początku,\\
    \textit{Start-Timestamp}.
    \item Wszelkie zmiany wykonywane są na lokalnych kopiach i zapisywane
    na końcu transakcji.
    \item Aktualizacje wykonywane przez inne transakcje nie są odczytywane.
    \item Jeśli $T_1$ jest gotowa do zatwierdzenia, otrzymuje
    \textit{Commit-Timestamp($T_1$)}, większy od wszystkich znaczników
    \textit{Start-Timestamp} i \textit{Commit-Timestamp}\\
    rozpoczętych i zakończonych już transakcji.
    \item Transakcja zostaje zatwierdzona tylko wówczas, jeśli żadna inna
    $T_2$
    z czasem zakończenia \textit{Commit-Timestamp($T_2$)} zawartym w
    przedziale\\
    \lbrack\textit{Start-Timestamp($T_1$)},~
    \textit{Commit-Timestamp($T_1$)}\rbrack~~nie zapisała danych,
    które zapisała również $T_1$.
    \item W przeciwnym wypadku $T_1$ zostaje wycofana.
    \item \textbf{\textit{First-commiter-wins} zapobiega lost update (P4).}
    \item Po zatwierdzeniu $T_1$, zmiany wykonane przez nią są widoczne
    przez wszystkie inne transakcje o czasie rozpoczecia wiekszym od
    \textit{Commit-Timestamp} transakcji $T_1$.
\end{itemize}

\textbf{SNAPSHOT ISOLATION} \textit{(First-writer-wins)}
\begin{itemize}
    \item Podobnie jak wcześniej, ale są stosowane
    \textbf{blokady do zapisu}.
    \item Przy każdym zapisie transakcja wykonuje podobne sprawdzenie jak
    \textit{First-commiter-wins} na końcu transakcji.
    \item Przechowywane są różne wersje danych.
    \item Transakcja czyta dane zatwierdzone przed początkiem transakcji.
    \item \textbf{Brak blokad do odczytu}, operacja odczytu nie blokuje
    zapisu ani innych operacji odczytu.
    \item \textbf{Długotrwałe blokady wyłącznie do zapisu}.
    \item $T_1$ przy każdym zapisie sprawdza, czy istnieje $T_2$, która
    zmodyfikowała dane zapisywane i zakończyła sie powodzeniem.
    \item Jeśli tak, $T_1$ jest wycofywana.
    \item W systemie MS SQL Server tak działa poziom SNAPSHOT.
\end{itemize}

% subsection poziomy_izolacji_transakcji (end)
\pagebreak

\subsection{Sterowanie współbieżnymi transakcjami w oparciu o blokady}
\label{sub:sterowanie_wspolbieznymi_blokady}

% subsection sterowanie_współbieżnymi_transakcjami_w_oparciu_o_blokady (end)

\subsection{Sterowanie współbieżnymi transakcjami z wykorzystaniem
wielowersyjności i blokad} % (fold)
\label{sub:sterowanie_wspolbieznymi_wielowier}

\horrule{0.5pt}
Dla przedstawionego harmonogramu proszę podać jak będzie wyglądać sterowanie
współbieżnością w wybranym przez egzaminatora poziomie izolacji
transakcji.\\
\horrule{0.5pt}

\pagebreak

\horrule{0.5pt}
Proszę \underline{omówić} wybrane przez egzaminatora \textbf{problemy}
związane ze współbieżnym wykonaniem transakcji.\\
\horrule{0.5pt}

\textbf{P0 Dirty Write} \textit{(nadpisanie brudnopisu)}
\begin{itemize}
    \item Transakcja $T_1$ modyfikuje daną.
    \item Transakcja $T_2$ dalej modyfikuje daną \textbf{zanim} $T_1$
    zostanie \textbf{zatwierdzona lub wycofana}.
    \item \texttt{WRITE\_1[x]}
    \item \texttt{WRITE\_2[x]}
    \item (\texttt{COMMIT\_1[x]} lub \texttt{ABORT\_1[x]}) i
    (\texttt{COMMIT\_2[x]} lub \texttt{ABORT\_2[x]})
    w dowolnej kolejności.
\end{itemize}

\textbf{P1 Dirty Read} \textit{(czytanie brudnopisu)}
\begin{itemize}
    \item Transakcja $T_1$ modyfikuje daną.
    \item Transakcja $T_2$ czyta daną \textbf{zanim} $T_1$ zostanie
    zatwierdzona lub wycofana.
    \item \textbf{Jeśli} $T_1$ zostaje \textbf{wycofana} $T_2$ przeczytało
    daną, która nie została zatwierdzona czyli w sumie nigdy nie
    istniała.
    \item \texttt{WRITE\_1[x]}
    \item \texttt{READ\_2[x]}
    \item (\texttt{COMMIT\_1[x]} lub \texttt{ABORT\_1[x]}) i
    (\texttt{COMMIT\_2[x]} lub \texttt{ABORT\_2[x]})
    w dowolnej kolejności.
\end{itemize}

\textbf{P2 Non-repeatable read}
\begin{itemize}
    \item Transakcja $T_1$ czyta daną.
    \item Transakcja $T_2$ modyfikuje lub usuwa daną oraz zostaje
    \textbf{zatwierdzona}.
    \item Jeśli $T_1$ spróbuje znowu przeczytać daną, otrzyma zmodyfikowaną
    wartość albo odkryje, że dana została skasowana.
    \item \texttt{READ\_1[x]}
    \item \texttt{WRITE\_2[x]}
    \item (\texttt{COMMIT\_1[x]} lub \texttt{ABORT\_1[x]}) i
    (\texttt{COMMIT\_2[x]} lub \texttt{ABORT\_2[x]})
    w dowolnej kolejności.
\end{itemize}

\textbf{P3 Phantoms}
\begin{itemize}
    \item Transakcja $T_1$ czyta zestaw danych spełniający jakiś
    \textit{predykat}.
    \item Transakcja $T_2$ \textbf{tworzy daną} spełniająca ten
    \textit{predykat} i zostaje zatwierdzona.
    \item Jeśli $T_1$ spróbuje znowu przeczytać zestaw danych z tym samym\\
    \textit{predykat}em otrzyma zestaw danych
    \textbf{inny od pierwotnego}.
    \item \texttt{READ\_1[P]}
    \item \texttt{WRITE\_2[y in P]}
    \item (\texttt{COMMIT\_1[x]} lub \texttt{ABORT\_1[x]}) i
    (\texttt{COMMIT\_2[x]} lub \texttt{ABORT\_2[x]})
    w dowolnej kolejności.
\end{itemize}

\textbf{P4 Lost Update}
\begin{itemize}
    \item Transakcja $T_1$ czyta element danych.
    \item Transakcja $T_2$ aktualizuje ten element i
    \textbf{zostaje zatwierdzona}.
    \item Transakcja $T_1$ aktualizuje ten sam element i
    \textbf{zostaje zatwierdzona}.
    \item \texttt{READ\_1[x]}
    \item \texttt{WRITE\_2[x]}
    \item \texttt{COMMIT\_2}
    \item \texttt{WRITE\_1[x]}
    \item \texttt{COMMIT\_1}
\end{itemize}

\textbf{P4C Lost Update} \textit{(dla operacji na kursorze)}
\begin{itemize}
    \item \texttt{READ\_CURSOR\_1[x]}
    \item \texttt{WRITE\_2[x]}
    \item \texttt{COMMIT\_2}
    \item \texttt{WRITE\_CURSOR\_1[x]}
    \item \texttt{COMMIT\_1}
\end{itemize}

\textbf{A5A Read Skew} \textit{(skrzywiony odczyt)}
\begin{itemize}
    \item Transakcja $T_1$ odczytuje \texttt{x}.
    \item Transakcja $T_2$ aktualizuje \texttt{x} oraz \texttt{y} i zostaje
    \textbf{zatwierdzona}.
    \item Jeśli $T_1$ odczyta \texttt{y} to będzie miała niespójny obraz
    danych.
    \item \texttt{READ\_1[x]}
    \item \texttt{WRITE\_2[x]}
    \item \texttt{WRITE\_2[y]}
    \item \texttt{COMMIT\_2}
    \item \texttt{READ\_1[y]}
    \item (\texttt{COMMIT\_1} lub \texttt{ABORT\_1})
\end{itemize}


\textbf{A5B Write Skew} \textit{(skrzywiony zapis)}\\\\
Załóżmy, że na elementy danych \texttt{x} oraz \texttt{y} narzucono pewne
ograniczenie \texttt{C()}. Każda transakcja z osobna dba o spełnienie
\texttt{C()}.
\begin{itemize}
    \item Transakcja $T_1$ odczytuje \texttt{x} (ew. też \texttt{y}).
    \item Transakcja $T_2$ odczytuje \texttt{y} (ew. też \texttt{x}).
    \item $T_1$ zapisuje \texttt{y} a $T_2$ zapisuje \texttt{x} i
    \textbf{obydwie zostają zatwierdzone}.
    \item Ostatnie cztery operacje mogą być zrealizowane w dowolnej
    (sensownej) kolejności.
    \item Każda transakcja przy zapisnie dba o spełnienie ograniczenia
    \texttt{C()}, jednak w wyniku przeplatanego wykonania ograniczenie
    \texttt{C()} może nie być spełnione po zatwierdzeniu obydwu
    transakcji.
    \item \texttt{READ\_1[x]}
    \item \texttt{READ\_2[y]}
    \item \texttt{WRITE\_1[y], WRITE\_2[x], COMMIT\_1, COMMIT\_2}
    w dowolnej\\ sensownej kolejności.
\end{itemize}

% subsection sterowanie_współbieżnymi_transakcjami (end)

\pagebreak

\subsection{Zakleszczenia} % (fold)
\label{sub:zakleszczenia}

\horrule{0.5pt}
Proszę \textbf{napisać przykładowy harmonogram}, który doprowadzi do
\textbf{zakleszczenia}.
Jak mogą być wykrywane zakleszczenia?\\
\horrule{0.5pt}

\begin{center}
    \begin{tabular}{|p{5cm}|p{5cm}|}
        \hline
        \textbf{$T_1$} & \textbf{$T_2$} \\
        \hline
        \texttt{read(K)} \textit{(zakłada blokadę S na K)} & \texttt{}\\
        % \texttt{K = 2 000,00} & \texttt{}\\
        \texttt{} & \texttt{read(K)} \textit{(zakłada blokadę S na K)}\\
        % \texttt{} & \texttt{K = 2 000,00}\\
        % \texttt{wybierz 300,00zł} & \texttt{}\\
        % \texttt{K := K - 300} \textit{(= 1 700)}& \texttt{}\\
        \texttt{write(K)} \textit{(żąda blokady X na K)} & \texttt{}\\
        % \texttt{wait} & \texttt{wybierz 100,00 zł}\\
        % \texttt{wait} & \texttt{K := K - 100} \textit{(= 1 900)}\\
        \texttt{wait} & \texttt{write(K)} \textit{(żąda blokady X na K)}\\
        \texttt{wait} & \texttt{wait}\\
        \texttt{wait} & \texttt{wait}\\
        \hline
    \end{tabular}
\end{center}


\textbf{WYKRYWANIE ZAKLESZCZEŃ}
\begin{itemize}
    \item \textbf{GRAF OCZEKIWANIA} \textit{(wait-for graph)}
    \begin{itemize}
        \item Każda wykonywana transakcja jest wierzchołkiem w grafie.
        \item Jeśli transakcja $T_i$ próbuje zablokować element danych,
        który \textbf{jest już blokowany} przez inną transakcję $T_k$
        z użyciem \textbf{konflikowej blokady}, w grafie tworzona
        jest krawędź skierowana z $T_i$ do $T_k$.
        \item Po zwolnieniu blokady krawędź jest usuwana.
        \item \textbf{Cykl w grafie oznacza zakleszczenie.}
        \item Wybór ofiary - na ofiare można wybrać transakcje młodszą, lub
        tę, która mniej zmodyfikowała (tę, której wycofanie jest
        prostsze).
    \end{itemize}

    \item \textbf{LIMITY CZASU} \textit{(timeouts)}
    \begin{itemize}
        \item Jeśli transakcja czeka na zasób \textbf{dłużej niż}
        przyjeta \textbf{wartość progowa}, to system przyjmuje, że
        uległa zakleszczeniu i \textbf{anuluje ją}, bez wzgledu na to
        czy zakleszczenie rzeczywiście wystąpiło, czy nie.
    \end{itemize}
\end{itemize}
% subsection zakleszczenia (end)

\pagebreak

\subsection{Kursory, sterowanie współbieżnością w kursorach} % (fold)
\label{sub:kursory_sterowanie_wspolbieznoscia_w_kursorach}

\horrule{0.5pt}
Proszę \underline{omówić}, jak wygląda \textbf{sterowanie współbieżnością w
\underline{kursorach}} w systemie Microsoft SQL Server.\\
\horrule{0.5pt}

\textbf{OPCJE PRZY OTWIERANIU KURSORA}
\begin{itemize}
    \item \texttt{READ\_ONLY}
    \begin{itemize}
        \item \textbf{Nie można} wykonywać pozycjonowanych zmian wierszy
        przez kursor.
        \begin{minted}{sql}
UPDATE ... WHERE CURRENT OF CursorName
        \end{minted}

        \item Blokady \textbf{nie są zakładane}.
    \end{itemize}

    \item \texttt{SCROLL LOCKS}
    \begin{itemize}
        \item Kursor jest otwarty w transakcji \textbf{jawnej}:
        \begin{itemize}
            \item Zakładne długotrwałe blokady \texttt{U} \textit{(update)}
            i blokady kursora \textit{(scroll locks)}.
            \item Blokady są zwalniane w momencie przejścia do innego
            wiersza.
        \end{itemize}

        \item Kursor jest otwarty \textbf{poza transakcją}:
        \begin{itemize}
            \item Zakładne są tylko blokady kursora.
        \end{itemize}

        \item Niewłączona opcja \textbf{automatycznego zamykania} kursorów
        na końcu transakcji może sprawić, że
        \textbf{blokady są trzymane} nadal po zakończeniu transakcji.

        \begin{minted}{sql}
SET CURSOR_CLOSE_ON_COMMIT ON
ALTER DATABASE SET CURSOR_CLOSE_ON_COMMIT
        \end{minted}
    \end{itemize}

    \item \texttt{OPTIMISTIC (WITH VALUES)}
    \begin{itemize}
        \item Przy \textbf{odczycie} wiersza \textbf{nie są zakładane}
        blokady.
        \item Przy próbie modyfikacji wiersza nastepuje
        \textbf{sprawdzenie, czy inna\\ transkacja tego nie zrobiła}
        \textit{(już po odczycie
        przez kursor, ale przed próbą modyfikacji)}.
        \item Wiersz wczytywany jest jeszcze raz i \textbf{porównywane są
        wartości} w kolumnach.
        \item Jeśli sie \textbf{nie zmieniły}, to aktualizacja nastepuje,
        jeśli nie, zgłaszany jest \textbf{błąd}.
    \end{itemize}

    \pagebreak
    \item \texttt{OPTIMISTIC (WITH ROW VERSIONING)}
    \begin{itemize}
        \item Podobnie, ale w tabeliu musi być kolumna typu
        \texttt{rowversion}\\
        \textit{(w MSSQL 2000 i 2005 timestamp)}.
        \item Wartość w takiej kolumnie jest zawsze automatycznie
        modyfikowana przy modyfikacji wiersza, nawet jeśli jest to
        modyfikacja typu \texttt{pole1 = pole1}.
    \end{itemize}

\end{itemize}
