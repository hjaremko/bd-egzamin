\section{Model ER}
\label{sec:model_er}

\horrule{0.5pt}
Proszę \underline{przedstawić przykład} diagramu \textbf{ER} w notacji
\textbf{Barkera}, zawierającego dwie encje i związek między nimi \textit{(związek
bez własnych atrybutów)}. Diagram powinien być taki, że po jego
transformacji do modelu relacyjnego \textbf{otrzymamy trzy relacje} \textit{(trzy
tabele)}.\\
\horrule{0.5pt}

% \includegraphics[scale=0.521]{erd}

\begin{center}
    \begin{tikzpicture}[node distance=1.5in]

        \matrix [entity=Pracownik] {
        \properties{
        +IdPracownika,
        *Nazwisko,
        *Etat,
        *Pensja
        }
        };

        \matrix [entity=Projekt, right=of Pracownik-+IdPracownika, entity anchor=Projekt-+NrProjektu]  {
        \properties{
        +NrProjektu,
        *Nazwa,
        *Sponsor
        }
        };

        \draw [many to many] (Pracownik-+IdPracownika)   to (Projekt-+NrProjektu);
    \end{tikzpicture}
\end{center}

% \vskip 0.1cm

% \begin{center}
% \entity{pracownik}[
%     \begin{tabular}{l}
%       Pracownik \\
%       \hline
%       \#~IdPracownika \\
%       \textasteriskcentered~Nazwisko \\
%       \textasteriskcentered~Etat \\
%       \textasteriskcentered~Pensja \\
%     \end{tabular}
% ]\hspace{2.5cm}
% \entity{projekt}[
%     \begin{tabular}{l}
%       Projekt \\
%       \hline
%       \#~NrProjektu \\
%       \textasteriskcentered~Nazwa \\
%       \textasteriskcentered~Sponsor \\
%     \end{tabular}
% ]
% \ncline[arrowscale=2]{>-<}{pracownik}{projekt}
% \naput[npos=0.1]{M}\naput[npos=0.85]{N}
% \end{center}
Po transformacji diagramu zostaną utworzone trzy tabele.
\begin{center}
    \begin{tabular}{cc}
        \begin{minipage}{.5\linewidth}
            \begin{center}
                \begin{tabular}{|l|}
                    \hline
                    \texttt{Pracownicy} \\
                    \hline
                    \texttt{\#~IdPracownika} \\
                    \texttt{*~Nazwisko} \\
                    \texttt{*~Etat} \\
                    \texttt{*~Pensja} \\
                    \hline
                \end{tabular}
            \end{center}

        \end{minipage} &
        \begin{minipage}{.5\linewidth}
            % \begin{center}
            \begin{tabular}{|l|}
                \hline
                \texttt{Projekty} \\
                \hline
                \texttt{\#~NrProjektu} \\
                \texttt{*~Nazwa} \\
                \texttt{*~Sponsor} \\
                \hline
            \end{tabular}
            % \end{center}
        \end{minipage}
    \end{tabular}
\end{center}
\begin{center}
    \begin{tabular}{|l|}
        \hline
        \texttt{Pracownicy\_Projekty} \\
        \hline
        \texttt{\#~IdPracownika REFERENCES Pracownicy(IdPracownika)} \\
        \texttt{\#~NrProjektu REFERENCES Projekty(NrProjektu)} \\
        \texttt{PRIMARY KEY (IdPracownika, NrProjektu)}\\
        \hline
    \end{tabular}
\end{center}

\horrule{0.5pt}
Proszę \underline{omówić} trzy schematy \textbf{transformacji hierarchii}
encji do modelu relacyjnego.\\
\horrule{0.5pt}
\begin{enumerate}
    \item Utworzenie \textbf{jednej tabeli ze wszystkimi} atrybutami i
    kluczami obcymi, tj. wspólnymi i specyficznymi dla podencji.
    \item Utworzenie \textbf{osobnej tabeli dla każdej podencji}.
    \item Utworzenie \textbf{osobnej tabeli na atrybuty wspólne i osobnej
    tabeli dla każdej podencji}.
\end{enumerate}
{\small
Tabele powstałe z podencji zawierają klucz podstawiowy i atrybuty
specyficzne, tabela wspólna i tabele powstałe z podencji są
połączone ograniczeniami referencyjnymi.
}
