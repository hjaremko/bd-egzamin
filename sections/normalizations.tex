\section{Architektury systemów baz danych}
\label{sec:architektury_systemow}

\section{Relacyjny model danych, normalizacja relacji}
\label{sec:relacyjny_model_danych_normalizacja_relacji}

\horrule{0.5pt}
Proszę \underline{podać przykład} tabeli (relacji), która \textbf{jest w
trzeciej} postaci normalnej, ale \textbf{nie jest} w postaci normalnej
\textbf{Boyce’a Codd’a}.\\
\horrule{0.5pt}\\

Relacja \texttt{\{Miasto, Ulica, Kod\}}\\
Klucze:
\begin{enumerate}
    \item \texttt{\{Miasto, Ulica\}}
    \item \texttt{\{Ulica, Kod\}}
\end{enumerate}

Dodatkowe zależności:
\begin{enumerate}
    \item \texttt{ \{Kod\} --> \{Miasto\}}\\
\end{enumerate}

\begin{tcolorbox}
    \textbf{Tylko 3NF} - \texttt{Kod} nie jest nadkluczem.
\end{tcolorbox}

% \pagebreak

\horrule{0.5pt}
Dla tabeli (relacji) z podanymi zależnościami funkcyjnymi \underline{proszę
powiedzieć} w której \textbf{postaci normalnej} jest ta tabela.\\
\horrule{0.5pt}\\

Relacja \texttt{\{CustomerID, Firstname, Surname, Telephone\}}\\
Klucze: \texttt{\{CustomerID, Telephone\}}\\
Dodatkowe zależności:
\begin{enumerate}
    \item \texttt{ \{CustomerID\} --> \{Firstname, Surname\}}
    \item \texttt{ \{Telephone\} --> \{CustomerID\}}
\end{enumerate}

\begin{tcolorbox}
    \textbf{Tylko 1NF} - Częściowa funkcyjna zależność atrybutu wtórnego od
    podzbioru właściwego klucza w zależności nr 2.
\end{tcolorbox}

\pagebreak

Relacja \texttt{\{Tournament, Year, Winner, Winner Date of Birth\}}\\
Klucze: \texttt{\{Tournament, Year\}}\\
Dodatkowe zależności:
\begin{enumerate}
    \item \texttt{ \{Winner\} --> \{Winner Date of Birth\}}
\end{enumerate}

\begin{tcolorbox}
    \textbf{Tylko 2NF} - Atrybut niekluczowy funkcyjnie zależny od innego
    atrybutu niekluczowego.
\end{tcolorbox}

\vskip 0.5cm
Relacja \texttt{\{PESEL, NrPłyty, Od, Do, NrDowodu\}}\\
Klucze:
\begin{enumerate}
    \item \texttt{\{PESEL, NrPłyty, Od\}}
    \item \texttt{\{NrDowodu, NrPłyty, Od\}}
\end{enumerate}

Dodatkowe zależności:
\begin{enumerate}
    \item \texttt{ \{NrDowodu\} --> \{PESEL\}}
\end{enumerate}

\begin{tcolorbox}
    \textbf{Tylko 3NF} - Lewa strona zależności nr 1 nie jest nadkluczem.
\end{tcolorbox}

\vskip 0.5cm
Relacja \texttt{\{Restaurant, Pizza Type, Delivery Area\}}\\
Klucze: \texttt{\{Restaurant, Pizza Type, Delivery Area\}}

Dodatkowe zależności:
\begin{enumerate}
    \item \texttt{ \{Restaurant\} ->> \{Pizza Type\}}
    \item \texttt{ \{Restaurant\} ->> \{Delivery Area\}}
\end{enumerate}

\begin{tcolorbox}
    \textbf{Tylko BCNF} - Jeśli założymy, że wszystkie typy pic są dowożone na
    każdy obszar to występują nietrywialne zależności wielowartościowe, których
    lewa strona nie zawiera klucza.
\end{tcolorbox}

\pagebreak

\horrule{0.5pt}
Podaną tabelę należy doprowadzić do
\textbf{postaci normalnej Boyce’a Codd’a}.\\
\horrule{0.5pt}

\vskip 0.5cm

\textbf{PRZYKŁAD 1.}

\vskip 0.5cm

\begin{adjustbox}{width=\columnwidth,center}
    % \begin{center}
    \begin{tabular}{|l|l|l|l|l|l|}
        \hline
        \textbf{Tytuł} &
        \textbf{Rok} &
        \textbf{Długość} &
        \textbf{TypFilmu} &
        \textbf{NazwaStudia} &
        \textbf{AdresStudia}\\
        \hline
        Gwiezdne Wojny &
        1977 &
        124 &
        kolor &
        Fox &
        Hollywood \\
        \hline
        Poteżne Kaczory &
        1991 &
        104 &
        kolor &
        Disney &
        Buena Vista \\
        \hline
        Świat Wayna &
        1992 &
        95 &
        kolor &
        Paramount &
        Hollywood \\
        \hline
        Rodzina Adamsów &
        1991 &
        102 &
        kolor &
        Paramount &
        Hollywood \\
        \hline
    \end{tabular}
    % \end{center}
\end{adjustbox}

\vskip 0.5cm

\textbf{Klucz:} \texttt{\{Tytuł, Rok\}}\\
\textbf{Zależność funkcyjna:} \texttt{\{NazwaStudia\} --> \{AdresStudia\}}\\

\textbf{Dekomponujemy schemat na dwa zbiory:}\\
\texttt{\{\underline{NazwaStudia}, AdresStudia\}}\\
\texttt{\{NazwaStudia, \underline{Tytuł, Rok}, Długość, TypFilmu\}}\\

\vskip 0.5cm

\textbf{PRZYKŁAD 2.}

\vskip 0.5cm

% \begin{adjustbox}{width=\columnwidth,center}
\begin{center}
    \begin{tabular}{|l|l|l|}
        \hline
        \textbf{Person} &
        \textbf{Shop Type} &
        \textbf{Nearest Shop}\\
        \hline
        Davidson & Optician & Eagle Eye \\
        \hline
        Davidson & Hairdresser & Snippets \\
        \hline
        Wright & Bookshop & Merlin Books \\
        \hline
        Fuller & Bakery & Doughy's \\
        \hline
        Fuller & Hairdresser & Sweeney Todd's \\
        \hline
        Fuller & Optician & Eagle Eye \\
        \hline
    \end{tabular}
\end{center}
% \end{adjustbox}

\vskip 0.5cm

\textbf{Klucze:} \texttt{\{Person, Shop Type\}; \{Person, Nearest Shop\}}\\
\textbf{Zależność funkcyjna:} \texttt{\{Nearest Shop\} --> \{Shop Type\}}\\

\textbf{Dekomponujemy schemat na dwa zbiory:}\\
\texttt{\{\underline{Nearest Shop}, Shop Type\}}\\
\texttt{\{\underline{Nearest Shop, Person}\}}\\

\pagebreak

\horrule{0.5pt}
Proszę \underline{podać przykład} uzasadniający \textbf{denormalizację}.\\
\horrule{0.5pt}\\

\texttt{\textbf{Adresy:}\\
\{\textbf{{\color{green}\faKey}NrPracownika}, Ulica, Miasto,
Województwo, KodPocztowy\}}\\

Zakładamy nastepujące zależności funkcyjne:
\begin{enumerate}
    \item \texttt{\{Ulica, Miasto, Województwo\} --> \{KodPocztowy\}}
    \item \texttt{\{KodPocztowy\} --> \{Miasto, Województwo\}}
\end{enumerate}

Relacja \texttt{Adresy} jest w 2NF, ale nie jest w 3NF.\\

\textbf{Rozwiązanie 1} (wynikające z zależności funkcyjnej 1.)\\

\texttt{\{{\color{green}\faKey}{\color{blue}\faKey}Ulica,
{\color{green}\faKey}Miasto, {\color{green}\faKey}Województwo,
{\color{blue}\faKey}KodPocztowy\}} (*)\\
\texttt{\{{\color{green}\faKey}NrPracownika, Ulica, Miasto, Województwo\}}
(**)\\

Relacja (*) jest w 3NF, ale nie jest BCNF.\\
Relacja (**) jest w BCNF.\\

Zostały zachowane zależności funkcyjne, ale jest redundancja.\\
Tabele bedą \underline{łączone ze sobą aż przez \textbf{trzy} pola}
(klucz obcy złożony).\\
Można tego uniknąć poprzez dodanie atrybutu \texttt{ IdAdresu } do
relacji (*) z zależnością funkcyjną \texttt{\{IdAdresu\} --> \{Ulica,
Miasto, Województwo\}} oraz wstawienie \texttt{ IdAdresu } zamiast
\texttt{\{Ulica, Miasto, Województwo\}} w relacji (*). Mielibyśmy:\\

\texttt{\{{\color{red}\faKey}IdAdresu,
{\color{green}\faKey}{\color{blue}\faKey}Ulica,
{\color{green}\faKey}Miasto, {\color{green}\faKey}Województwo,
{\color{green}\faKey}KodPocztowy\}}
\textit{(tylko 3NF)}\\
\texttt{\{{\color{green}\faKey}NrPracownika, IdAdresu\}} \textit{(BCNF)}\\

Wróćmy jednak do rozwiązania bez \texttt{IdAdresu}. W pierwszej relacji jest
redundancja. Można dekomponować pierwszą relacje na dwie, doprowadzają do
BCNF:\\
\texttt{\{Miasto, Województwo, {\color{blue}\faKey}KodPocztowy\}} oraz\\
\texttt{\{{\color{green}\faKey}Ulica, {\color{green}\faKey}KodPocztowy\}}\\

Ostatecznie otrzymamy trzy relacje:\\
\texttt{\{Miasto, Województwo, {\color{blue}\faKey}KodPocztowy\}}\\
\texttt{\{{\color{green}\faKey}Ulica, {\color{green}\faKey}KodPocztowy\}}\\
\texttt{\{{\color{green}\faKey}NrPracownika, Ulica, Miasto, Województwo\}}\\

\textbf{"Zgubiliśmy" zależność funkcyjną nr 1}. Ponadto uzyskanie kodu
pocztowego\\ pracownika \underline{wymaga złączenia trzech tabel}.\\

W przypadku wersji z atrybutem \texttt{ IdAdresu } otrzymalibyśmy:

\texttt{\{Miasto, Województwo, {\color{blue}\faKey}KodPocztowy\}}\\
\texttt{\{{\color{red}\faKey}IdAdresu, {\color{blue}\faKey}Ulica,
{\color{blue}\faKey}KodPocztowy\}}\\
\texttt{\{{\color{green}\faKey}NrPracownika, IdAdresu\}}\\

\textbf{"Zgubiliśmy" zależność funkcyjną nr 1}. Uzyskanie miasta i województwa
pracownika \underline{wymaga złączenia trzech tabel}.\\

\textbf{Rozwiązanie 2} (wynikające z zależności funkcyjnej 1.)\\

\texttt{\{{\color{green}\faKey}KodPocztowy, Miasto, Województwo\}}\\
\texttt{\{{\color{green}\faKey}NrPracownika, Ulica, KodPocztowy\}}\\

Obie relacje są w BCNF.\\
Obydwie wcześniej przedstawione \textbf{zależności funkcyjne zostały
"zgubione"}.
Połączenie miedzy tabelami bedzie realizowane przez jedno pole.\\\\

Obydwa rozwiązania mają swoje wady i zalety. Zalety i wady ma również
wyjściowa relacja \texttt{Adresy} (chociaż jest tylko w 2NF, nie jest w
3NF).

\pagebreak

\horrule{0.5pt}
Proszę \underline{podać przykład} tabeli, która jest w postaci normalnej
\textbf{Boyce’a Codd’a}, a w której jest \textbf{redundancja}.\\
\horrule{0.5pt}

\vskip 0.5cm

% \begin{adjustbox}{width=\columnwidth,center}
\begin{center}
    \begin{tabular}{|l|l|l|}
        \hline
        \textbf{Nr Pracownika} &
        \textbf{Język Programowania} &
        \textbf{Język Obcy}\\
        \hline
        1 & C & niemiecki \\
        \hline
        1 & Java & angielski \\
        \hline
        2 & C & francuski \\
        \hline
        2 & Java & niemiecki \\
        \hline
        2 & & hiszpański \\
        \hline
    \end{tabular}
\end{center}
% \end{adjustbox}

\vskip 0.5cm

\textbf{Klucz:}\texttt{\{Nr Pracownika, Język Programowania, Język Obcy\}}\\
\textbf{Zależności:}
\begin{enumerate}
    \item \texttt{\{Nr Pracownika\} ->> \{Język Programowania\}}
    \item \texttt{\{Nr Pracownika\} ->> \{Język Obcy\}}
\end{enumerate}

\begin{tcolorbox}
    \textbf{Tylko BCNF} - Występują zależności wielowartościowe, których lewa
    strona nie zawiera klucza.
\end{tcolorbox}

% \pagebreak

\horrule{0.5pt}
Proszę \underline{podać przykład} tabeli, która jest w \textbf{5NF},
ale jest w niej \textbf{redundancja}.\\
\horrule{0.5pt}

\vskip 0.5cm

% \begin{adjustbox}{width=\columnwidth,center}
\begin{center}
    \begin{tabular}{|l|l|l|}
        \hline
        \textbf{Producent} &
        \textbf{Wyrób} &
        \textbf{Część}\\
        \hline
        1 & 1 & 1 \\
        \hline
        1 & 2 & 1 \\
        \hline
    \end{tabular}
\end{center}
% \end{adjustbox}

\vskip 0.5cm

\textbf{Klucz:} \texttt{\{Producent, Wyrób, Część\}}\\

\begin{tcolorbox}
    \textbf{5CNF} - Zakładamy, że \texttt{Producent} może produkować daną część
    tylko do jednego wyrobu, nawet jeśli mogłaby być użyta w innym. Zatem
    \textbf{brak zależności wielowartościowych oraz klucz obejmuje wszystkie
    atrybuty} a wciąż jest redundancja (powtarzamy, że \texttt{Producent} 1
    produkuje \texttt{Część} 1).
\end{tcolorbox}

\pagebreak

\horrule{0.5pt}
Proszę \underline{przedstawić} algorytm doprowadzenia relacji do
\textbf{postaci normalnej Boyce’a Codd’a}.\\
\horrule{0.5pt}

\begin{itemize}
    \item Szukamy wszystkich \underline{nietrywialnych, pełnych zależności
    funkcyjnych}, które naruszają warunek BCNF, tzn.
    \textbf{lewa strona} zalezności funkcyjnej
    \textbf{nie jest nadkluczem}.
    \item Bierzemy jedną z takich zależności funkcyjnych $A \rightarrow B$
    (obojetnie którą, algorym jest niederministyczny).
    \item Dzielimy schemat relacji na dwa nierozłączne podzbiory:
    jeden zawierający wszystkie atrybuty wystepujące w zależności (*)
    naruszającej BCNF,
    drugi zawierający atrybuty z lewej strony rozważanej zależności
    (*) oraz atrybuty nie wystąpujące ani z lewej ani z prawej strony
    tej zależności.
    \item Strategie stosujemy do relacji powstałych w wyniku dekompozycji
    do chwili, gdy wszystkie relacje są w BCNF.
\end{itemize}
