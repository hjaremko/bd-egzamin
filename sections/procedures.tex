\section{Procedury, funkcje i wyzwalacze}
\label{sec:procedury_funkcje_i_wyzwalacze}

\horrule{0.5pt}
Proszę \underline{napisać} \textbf{procedurę} w języku Transact SQL,
funkcjonalność procedury poda egzaminator.\\
\horrule{0.5pt}

\begin{minted}{sql}
CREATE PROCEDURE ProcName (@Arg TYPE, @OutArg TYPE OUTPUT)
AS
BEGIN
    ...
END
\end{minted}

\horrule{0.5pt}
Proszę \underline{napisać} \textbf{funkcję skalarną} w języku Transact SQL,
opisaną przez egzaminatora.\\
\horrule{0.5pt}

\begin{minted}{sql}
CREATE FUNCTION FuncName (@Arg TYPE, @OutArg TYPE OUTPUT)
RETURNS TYPE
AS
BEGIN
    ...
END
\end{minted}

\horrule{0.5pt}
Proszę \underline{napisać} w języku Transact SQL opisaną przez egzaminatora
\textbf{funkcję} zwracającą zestaw rekordów.\\
\horrule{0.5pt}

\begin{minted}{sql}
CREATE FUNCTION ProcName (@Arg TYPE, @OutArg TYPE OUTPUT)
RETURNS @TableName TABLE (
    ColumnName1 TYPE,
    ColumnName2 TYPE
)
AS
BEGIN
    ...
END
---------------------------------------------------------
CREATE FUNCTION ProcName (@Arg TYPE, @OutArg TYPE OUTPUT)
RETURNS TABLE
AS
BEGIN
    RETURN (SELECT ...)
END
\end{minted}

\horrule{0.5pt}
Proszę \underline{napisać} w języku Transact SQL \textbf{wyzwalacz}, który
zrealizuje zadaną przez egzaminatora funkcjonalność.\\
\horrule{0.5pt}

\begin{minted}{sql}
CREATE TRIGGER TriggerName
ON TableName
AFTER INSERT, UPDATE, DELETE
--INSTEAD OF
AS
BEGIN
    ...
END
\end{minted}
