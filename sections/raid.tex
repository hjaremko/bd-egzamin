\section{Budowa fizyczna baz danych - macierze RAID}
\label{sec:budowa_fizyczna_baz_danych_macierze_raid}

\horrule{0.5pt}
Proszę \underline{omówić} macierze \textbf{RAID} \textit{(wybrany przez
egzaminatora poziom)}.\\
\horrule{0.5pt}

\textbf{RAID 0}
\begin{itemize}
    \item Dane umieszczane są równomierne na dwóch lub więcej dyskach.
    \item Jeden dysk: \texttt{B1, B2, B3, B4, B5, B6, B7, B8, ...}
    \item Cztery dyski (RAID 0) - przykład.

        \begin{center}
        \begin{tabular}{cccc}
            \begin{minipage}{2cm}

            \begin{center}
                \begin{tabular}{|c|}
                  \hline
                  \texttt{B1} \\
                  \hline
                  \texttt{B5} \\
                  \hline
                  \texttt{...} \\
                  \hline
                \end{tabular}
            \end{center}

            \end{minipage} &
            \begin{minipage}{2cm}

            \begin{center}
                \begin{tabular}{|c|}
                  \hline
                  \texttt{B2} \\
                  \hline
                  \texttt{B6} \\
                  \hline
                  \texttt{...} \\
                  \hline
                \end{tabular}
            \end{center}

            \end{minipage} &
            \begin{minipage}{2cm}

            \begin{center}
                \begin{tabular}{|c|}
                  \hline
                  \texttt{B3} \\
                  \hline
                  \texttt{B7} \\
                  \hline
                  \texttt{...} \\
                  \hline
                \end{tabular}
            \end{center}

            \end{minipage} &
            \begin{minipage}{2cm}

            \begin{center}
                \begin{tabular}{|c|}
                  \hline
                  \texttt{B4} \\
                  \hline
                  \texttt{B8} \\
                  \hline
                  \texttt{...} \\
                  \hline
                \end{tabular}
            \end{center}

            \end{minipage}
        \end{tabular}
        \end{center}

    \item \textbf{ZALETY}
    \begin{itemize}
        \item \textbf{Szybszy odczyt i zapis} w porównaniu z pojedynczym
              dyskiem dzięki operacjom równoległym.
        \item Był używany do połączenia dwóch lub więcej mniejszych dysków
              w jeden większy logiczny dysk.
    \end{itemize}

    \item \textbf{WADY}
    \begin{itemize}
        \item \textbf{Brak odporności na błedy} - nie ma nadmiarowości!
        \item Niezawodność jest odwrotnie proporcjonalna do liczby dysków
              w systemie RAID 0.\\
              Niezawodność dwóch dysków jest połowę mniejsza od
              niezawodności jednego dysku.
        \item RAID 0 nie jest polecany w środowiskach podwyższonej
              odporności na błędy \textit{(mission-critical environments)}.
    \end{itemize}

\end{itemize}

\pagebreak

\textbf{RAID 1}
\begin{itemize}
    \item \textit{Mirroring} lub \textit{duplexing} (gdy każdy dysk ma
          osobny kontroler).
    \item Na ogół zawiera dwa dyski, czasem wiecej.
    \item Obydwa dyski mają taką samą zawartość.

        \begin{center}
        \begin{tabular}{cc}
            \begin{minipage}{2cm}

            \begin{center}
                \begin{tabular}{|c|}
                  \hline
                  \texttt{B1} \\
                  \hline
                  \texttt{B2} \\
                  \hline
                  \texttt{...} \\
                  \hline
                \end{tabular}
            \end{center}

            \end{minipage} &
            \begin{minipage}{2cm}

            \begin{center}
                \begin{tabular}{|c|}
                  \hline
                  \texttt{B1} \\
                  \hline
                  \texttt{B2} \\
                  \hline
                  \texttt{...} \\
                  \hline
                \end{tabular}
            \end{center}

            \end{minipage}
        \end{tabular}
        \end{center}

    \item \textbf{ZALETY}
    \begin{itemize}
        \item \textbf{Odczyt} jest prawie dwukrotnie \textbf{szybszy} w
              porównaniu z pojedynczym dyskiem.
        \item \textbf{Odporny na awarie} - $N$ dysków może przetrwać
              jednoczesną awarię $N - 1$ dysków.
    \end{itemize}

    \item \textbf{WADY}
    \begin{itemize}
        \item \textbf{Wolniejszy zapis} w porównaniu z pojedynczym
              dyskiem.\\
              Dane muszą być zapisane na obydwu dyskach, a na początku
              operacji głowice są na ogół nad innymi ścieżkami i sektorami.
        \item Pamieć podręczna powinna być włączona w celu przyśpieszenia
              operacji zapisu.
        \item Używa efektywnie jedynie 50\% całkowitej pojemności.
    \end{itemize}

    \item \textbf{TYPOWE ZASTOSOWANIE}
    \begin{itemize}
        \item \textbf{Dziennik transakcji}.
        \item \textbf{System operacyjny}.
        \item Dziennik transakcji jest zapisywany sekwencyjnie, najlepiej
              stosować jeden system RAID 1 dla każdego dziennika.
        \item Dziennik jest jednym z najważniejszych komponentów systemu
              baz danych, dlatego dysk z dziennikiem powinien być odporny
              na awarie.
    \end{itemize}

\end{itemize}

\pagebreak

\textbf{RAID 5}
\begin{itemize}
    \item Zawiera \textbf{trzy} lub więcej dysków.
    \item Zapisy są wykonywane blokami na wielu dyskach z wykorzystaniem
          bloków parzystości.
    \item Bloki mogą być wieksze niż sektory (np. 256 sektorów).
\end{itemize}

        \begin{center}
        \begin{tabular}{cccc}
            \begin{minipage}{2.6cm}

            \begin{center}
                \begin{tabular}{|c|}
                  \hline
                  \texttt{B1} \\
                  \hline
                  \texttt{B5} \\
                  \hline
                  \texttt{B8} \\
                  \hline
                  Parity {11,12,13} \\
                  \hline
                \end{tabular}
            \end{center}

            \end{minipage} &
            \begin{minipage}{2.5cm}

            \begin{center}
                \begin{tabular}{|c|}
                  \hline
                  \texttt{B2} \\
                  \hline
                  \texttt{B6} \\
                  \hline
                  Parity {8,9,10} \\
                  \hline
                  \texttt{B11} \\
                  \hline
                \end{tabular}
            \end{center}

            \end{minipage} &
            \begin{minipage}{2.5cm}

            \begin{center}
                \begin{tabular}{|c|}
                  \hline
                  \texttt{B3} \\
                  \hline
                  Parity {5,6,7} \\
                  \hline
                  \texttt{B9} \\
                  \hline
                  \texttt{B12} \\
                  \hline
                \end{tabular}
            \end{center}

            \end{minipage} &
            \begin{minipage}{2.5cm}

            \begin{center}
                \begin{tabular}{|c|}
                  \hline
                  {Parity 1, 2, 3} \\
                  \hline
                  \texttt{B7} \\
                  \hline
                  \texttt{B10} \\
                  \hline
                  \texttt{B13} \\
                  \hline
                \end{tabular}
            \end{center}

            \end{minipage}
        \end{tabular}
        \end{center}

\begin{itemize}
    \item \textbf{ZALETY}
    \begin{itemize}
        \item \textbf{Szybki odczyt}.
        \item Względnie efektywne wykorzystanie przestrzeni dyskowej.
        \item \textbf{Odporność na błedy}.
        \begin{itemize}
            \item Bloki parzystości są odczytywane jeśli przy odczycie
                  wykryty jest błąd sumy kontrolnej \textit{(CRC error)}.
            \item W takim przypadku pozostałe bloki z paska są automatycznie
                  \\użyte do odtworzenia informacji w bloku uszkodzonym.
            \item Podobnie dzieje się przy uszkodzeniu całego dysku.
            \item Komputer może być "nieświadomy" awarii dysku.\\
                  System RAID pracuje, chociaż nieco wolniej.
        \end{itemize}
    \end{itemize}

    \item \textbf{WADY}
    \begin{itemize}
        \item \textbf{Wolne operacje zapisu.}
        \begin{itemize}
            \item Jeśli blok ma być zapisany, system musi wykonać dwa
                  odczyty i dwa zapisy zamiast jednego zapisu.
            \item Zmieniany blok i odpowiedni blok parzystości musi być
                  odczytany, trzeba zapisać nową wartość w bloku
                  parzystości.
            \item Wystarczy znać różnice między starą i nową wartością
                  zmienianego bloku i starą wartość bloku parzystości.
            \item Na końcu nowy blok i zmieniony blok parzystości muszą
                  być zapisane na dysk.
        \end{itemize}
        \item Należy włączyć pamięć podręczną jeśli jest podtrzymywanie
              bateryjne. Operacje zapisu mogą być wówczas przyśpieszone.
        \item Wykorzystuje $\frac{1}{n}$ pojemności dysku na bloki
              parzystości, gdzie $n$ oznacza liczbę dysków w systemie.
    \end{itemize}

    \item \textbf{TYPOWE ZASTOSOWANIE}
    \begin{itemize}
        \item Systemy, w których
              \textbf{większość operacji to operacje odczytu}.
        \item \textbf{Tablice lub indeksy}, które są tylko do odczytu lub
              są rzadko modyfikowane.
        \item RAID 5 nie jest na ogół dobrym rozwiązaniem, jeśli więcej niż
              10 procent operacji to operacje zapisu.\\
              Należy jednak znać wydajność systemu po włączeniu buforowania.
    \end{itemize}
\end{itemize}

\textbf{RAID 10 (1+0)}
\begin{itemize}
    \item Poziomy RAID mogą być zagnieżdżane.\\
          Jeden system RAID może użyć innego systemu RAID jako elementu
          składowego zamiast pojedynczego dysku.
    \item RAID 10 jest systemem z przeplotem z systemami RAID 1 jako
          elementami składowymi \textit{(a stripe of mirrors)}.
    % \item \textbf{Jeden dysk z każdego systemu} RAID 1 może ulec awarii
    \item $N - 1$ \textbf{z każdego systemu} RAID 1 może ulec awarii
          bez całkowitej utraty danych.
\end{itemize}

\begin{center}
\begin{tikzpicture}
\Tree
[.RAID~0
    [.RAID~1
        [.500~GB ]
        [.500~GB ]
    ]
    [.RAID~1
        [.500~GB ]
        [.500~GB ]
    ]
    [.RAID~1
        [.500~GB ]
        [.500~GB ]
    ]
]
\end{tikzpicture}
\end{center}

\begin{itemize}
    \item Ten system może przetwać równoczesną awarię \textbf{do trzech}
          dysków.\\

    \item \textbf{CECHY}
    \begin{itemize}
        \item \textbf{Najszybszy zapis i odczyt}.
        \item Odporność na błędy.
        \item Wykorzystuje efektywnie 50\% pojemności całkowitej.
        \item Najlepszy, ale najdroższy.
        \item Polecany w systemach baz danych, gdy
              \textbf{liczba operacji zapisu}
              jest \\większa niż 10\% liczby wszystkich operacji.
    \end{itemize}
\end{itemize}

\textbf{RAID 01 (0+1)}
\begin{itemize}
    \item Różnice między RAID 0+1 i RAID 1+0 stanowi położenie każdego z
          systemów RAID.
    \item Jest uważany za \textbf{gorszy niż RAID 10}.
    \item \textbf{Nie przetrwa dwóch równoczensnych awarii}
          jeśli nie są to dyski z tego samego układu RAID.
    \item Po awarii jednego dysku, wszystkie dyski w drugim pasku stanowią
          krytyczne punkty. Połowa dysków przestaje być wykorzystywana.\\
          W systemie RAID 10 jeśli uszkodzeniu ulegnie jeden dysk, tylko
          jeden staje się krytycznym punktem układu.
\end{itemize}

\begin{center}
\begin{tikzpicture}
\Tree
[.RAID~1
    [.RAID~0
        [.500~GB ]
        [.500~GB ]
        [.500~GB ]
    ]
    [.RAID~0
        [.500~GB ]
        [.500~GB ]
        [.500~GB ]
    ]
]
\end{tikzpicture}
\end{center}

\textbf{WAŻNE UWAGI}
\begin{itemize}
    \item RAID nie powinien zastąpić odpowiedniej strategii robienia kopii
          zapasowych.
    \item Administrator musi wykonywać okresowo kopie zapasowe (pełne,
          różnicowe, plików, dziennika transakcji).
\end{itemize}
